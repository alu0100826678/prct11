\documentclass{beamer}        %Para crear un documento en Beamer.
\usepackage[spanish]{babel}  %Para que entienda en Español.
\usepackage[utf8]{inputenc}   %Para que acepte los acentos.
\usepackage{graphicx}   %Para añadir los gráficos.


%A continuación,indicamos cual será el título,el autor y la fecha:
\title[T.E]{Primera Práctica con Beamer}           
\author[Y.A.R]{Yoselin Armas Ramos}               %Lo que hemos escrito entre corchetes se escribe en la parte baja de la presentación.
\date[25/04/14]{25 de Abril de 2014} 
%%%%%%%%%%%%%%%%%%%%%%%%%%%%%%%%%%%%%%%%%%%%%%%%%%%%%%%%%%%%%%%%%%%%%%%%%%%%%%%%%%%%%%%%%%%%%%%%%
\usetheme{Madrid}
%%%%%%%%%%%%%%%%%%%%%%%%%%%%%%%%%%%%%%%%%%%%%%%%%%%%%%%%%%%%%%%%%%%%%%%%%%%%%%%%%%%%%%%%%%%%%%
%Creamos el documento y la portada:
\begin{document}  %Empieza el documento.

\begin{frame}  %Creamos la primera diapositiva.

  %\includegraphics[width=0.15\textwidth]{img/ullesc}   %Añadimos los logos.
  \hspace*{7.0cm}
  %\includegraphics[width=0.16\textwidth]{img/fmatesc}
  \titlepage  %Para que se cree la portada.

\end{frame}  %Termina la primera diapositiva.
%%%%%%%%%%%%%%%%%%%%%%%%%%%%%%%%%%%%%%%%%%%%%%%%%%%%%%%%%%%%%%%%%%%%
%%\begin{frame}
  %%\frametitle{Índice}
  %%\tableofcontents[pausesections]
%%\end{frame}
%%%%%%%%%%%%%%%%%%%%%%%%%%%%%%%%%%%%%%%%%%%%%%%%%%%%%%%%%%%%%%%%%%%%%%%%%%%%%%%%%%%%%%%%
\section{El número PI} %Creamos la primera sección:


\begin{frame}         %Comienza la diapositiva.

\frametitle{Introducción}   %Introducimos el título de la diapostitiva.
El número $\pi$ es un número irracional y una de las constantes matemáticas más importantes. 
Se emplea frecuentemente en Matematicas,Física e Ingeniería. El valor aproximado de $\pi$ es 3'14159265358979323846...
Es importante mencionar que $\pi$ es la relación entre la longitud de una circunferencia y su diámetro.
\subsection{Aproximación del número $\pi$}
Según Newton el número $\pi$ es igual a:
\begin{displaymath}
\arcsin \left(\frac{1}{2}\right) = \frac{\pi}{6}  
\end{displaymath}

\end{frame}
%%%%%%%%%%%%%%%%%%%%%%%%%%%%%%%%%%%%%%%%%%%%%%%%%%%%%%%%%%%%%%%%%%%%%%%%%%%%%%%%%%%%%%%%%%%

\section{Datos de interés} %Creamos la segunda sección:


\begin{frame}

\frametitle{Operaciones con el número $\pi$}
Hay diferentes fórmulas que incluyen datos o tienen que ver con el número $\pi$.
Por ejemplo: 

-Leibniz calculó de una forma más complicada en 1682 la siguiente serie matemática que lleva su nombre:
\begin{displaymath}
   \sum_{n=0}^{\infty} \frac{(-1)^n}{2n+1} = 1 - \frac{1}{3} + \frac{1}{5} - \dots = \frac{\pi}{4} 
\end{displaymath}

-Identidad de Euler
\begin{displaymath}
    e^{\pi*i} + 1 = 0 
\end{displaymath}
\subsection{Más fórmulas}
-Área bajo la campana de Gauss:
\begin{displaymath}
    \int_{-\infty}^{\infty} e^{-x^2} dx = \sqrt{\pi} 
\end{displaymath}

-Desarrollo en series:
\begin{displaymath}
   \pi = \sum_{k=0}^\infty \frac{2(-1)^k\; 3^{\frac{1}{2} - k}}{2k+1} 
\end{displaymath}
\end{frame}

%%%%%%%%%%%%%%%%%%%%%%%%%%%%%%%%%%%%%%%%%%%%%%%%%%%%%%%%%%%%%%%%%%%%%%%%%%%%%%%%%%%%%%%%%%%%%
\section{Bibliografía}%Creamos una sección nueva para la bibliografía

\begin{frame}
  \frametitle{Bibliografía}
 
  \begin{thebibliography}{10}

  \beamertemplatebookbibitems
  \bibitem[Plan Estudios,2011]{plan}
   Documento de vetificación del grado.
   (2011)

   \beamertemplatebookbibitems
   \bibitem[Guía Docente, 2013]{guía}
    Guía Docente.
    (2013)
    {small http://eguia.ull.es/matematicas.query.php?codigo=299341201}

    \beamertemplatebookbibitems
    \bibitem[URL: wikipedia]{latex}
     Wikipedia. {\smallhttp://es.wikipedia.org/wiki/Número\_$\pi$}

\end{thebibliography}
\end{frame}
%%%%%%%%%%%%%%%%%%%%%%%%%%%%%%%%%%%%%%%%%%%%%%%%%%%%%%%%%%%%%%%%%%%%%%%%%%%%%%%%%%%%
\end{document}    %Termina el documento.

